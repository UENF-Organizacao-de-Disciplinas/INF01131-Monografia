
    \chapter{Conclusão}
    
    Esta monografia teve como objetivo desenvolver um modelo de requisitos de segurança robusto para aplicações bancárias no Android, utilizando como base os principais desafios, ameaças, melhores práticas e recomendações para mitigação encontradas na literatura. Verificou-se que garantir a segurança em aplicativos bancários móveis é uma área complexa que exige uma abordagem multifacetada para proteger os dados do usuário e manter a integridade das transações financeiras.

    Durante a pesquisa, várias dificuldades foram encontradas, impactando o desenvolvimento do trabalho, especialmente em relação à obtenção, filtragem e análise de informações. A rápida evolução das ameaças e das tecnologias de segurança em aplicações bancárias móveis tornou desafiador manter uma revisão bibliográfica atualizada. Além disso, a grande quantidade de estudos listados com as palavras chaves pesquisadas dificultou a filtragem de informações relevantes e a identificação de padrões consolidados e práticas recomendadas. Essas dificuldades exigiram uma abordagem meticulosa e adaptável na análise das informações disponíveis, além de um esforço adicional na busca por fontes alternativas e confiáveis para fundamentar o trabalho.

    O principal resultado desta pesquisa é o modelo de requisitos de segurança fornecido para fomentar a segurança da informação dos aplicativos bancários Android.

    Primeiramente, a fase de elicitação de requisitos permitiu uma compreensão aprofundada das ameaças e vulnerabilidades específicas associadas a aplicações bancárias móveis. Através da análise da literatura e do estudo de casos de uso indevido, foram identificados os principais pontos críticos que precisavam ser abordados e utilizados ao definir, especificar e modelar os requisitos de segurança. Esta etapa foi essencial para garantir que o modelo de segurança desenvolvido fosse alinhado com as necessidades reais e práticas do setor bancário. 
    
    Essas contribuições estabeleceram um modelo de segurança abrangente, focado nos pilares da confidencialidade, integridade, autenticação e autorização. O modelo desenvolvido assegura que todas as comunicações e dados dentro da aplicação bancária sejam protegidos contra acessos não autorizados, garantindo que apenas entidades legítimas possam acessar informações sensíveis. A integridade é mantida através de verificações rigorosas e mecanismos de validação, prevenindo alterações não autorizadas nos dados. A autenticação robusta e a autorização rigorosa garantem que somente usuários e processos devidamente verificados possam realizar ações críticas na aplicação. 

    Por fim, a avaliação do modelo de requisitos de segurança foi realizada através de um estudo de caso aplicado à aplicação bancária Herd Financial e por um estudo de adequação a certificação de segurança financeira PCI DSS. Estes estudos envolveram a implementação teórica dos requisitos propostos a fim de avaliar a viabilidade e aplicabilidade do modelo proposto e a sua adequação a certificação financeira renomada para avaliar sua eficácia e robustez. Dessa forma, oferecendo um caminho claro para desenvolvedores que buscam fortalecer a segurança de suas aplicações bancárias no Android.
    
    Assim, o trabalho proporciona uma base sólida e confiável, crucial para o desenvolvimento de aplicações bancárias móveis seguras no Android, abordando as necessidades fundamentais de segurança no cenário de ameaças cibernéticas em constante evolução.

    Entretanto, a sofisticação crescente dos ataques requer uma colaboração entre a indústria e a academia a fim de encontrar soluções inovadoras e uma abordagem proativa à segurança. Uma direção futura para o tema é a combinação de outras tecnologias à segurança da informação, como a inteligência artificial citada por  \citeonline{Oguntimilehin2022}, o IOT citado por \citeonline{Krishna2021}, e a implementação de novas formas de autenticação como dito por \citeonline{Hussein2022}; e como elas podem fornecer uma defesa eficaz contra essas ameaças.