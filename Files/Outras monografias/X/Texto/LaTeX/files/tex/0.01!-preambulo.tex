% !TEX root = ./base.tex

%% abtex2-modelo-trabalho-academico.tex, v-1.9.5 laurocesar
%% Copyright 2012-2015 by abnTeX2 group at https://www.abntex.net.br
%%% This work may be distributed and/or modified under the
%% conditions of the LaTeX Project Public License, either version 1.3
%% of this license or (at your option) any later version.
%% The latest version of this license is in
%%  https://www.latex-project.org/lppl.txt
%% and version 1.3 or later is part of all distributions of LaTeX
%% version 2005/12/01 or later.
%%% This work has the LPPL maintenance status `maintained'.
%%
%% The Current Maintainer of this work is the abnTeX2 team, led
%% by Lauro César Araujo. Further information are available on
%% https://www.abntex.net.br/
%%% This work consists of the files abntex2-modelo-trabalho-academico.tex,
%% abntex2-modelo-include-comandos and abntex2-modelo-references.bib
%
% ------------------------------------------------------------------------
% ------------------------------------------------------------------------
% abnTeX2: Modelo de Trabalho Academico (tese de doutorado, dissertacao de
% mestrado e trabalhos monograficos em geral) em conformidade com
% ABNT NBR 14724:2011: Informacao e documentacao - Trabalhos academicos -
% Apresentacao
% ------------------------------------------------------------------------
% ------------------------------------------------------------------------

\documentclass[
  % -- opções da classe memoir --
  12pt,                 % tamanho da fonte
  openright,            % capítulos começam em pág ímpar (insere página vazia caso preciso)
  oneside,              % para impressão apenas no verso. Oposto a twoside
  a4paper,              % tamanho do papel.
  % twoside,            % para impressão em verso e anverso. Oposto a oneside
  % -- opções da classe abntex2 --
  chapter=TITLE,       % títulos de capítulos convertidos em letras maiúsculas
  %section=TITLE,       % títulos de seções convertidos em letras maiúsculas
  %subsection=TITLE,    % títulos de subseções convertidos em letras maiúsculas
  %subsubsection=TITLE, % títulos de subsubseções convertidos em letras maiúsculas
  % -- opções do pacote babel --
  english, % idioma adicional para hifenização
  french,  % idioma adicional para hifenização
  spanish, % idioma adicional para hifenização
  brazil   % o último idioma é o principal do documento
]{abntex2}

% --- Pacotes básicos ---
\usepackage{lmodern}        % Usa a fonte Latin Modern
\usepackage[T1]{fontenc}    % Selecao de codigos de fonte.
\usepackage[utf8]{inputenc} % Codificacao do documento (conversão automática dos acentos)
\usepackage{lastpage}       % Usado pela Ficha catalográfica
\usepackage{indentfirst}    % Indenta o primeiro parágrafo de cada seção.
\usepackage{color}          % Controle das cores
\usepackage{graphicx}       % Inclusão de gráficos
\usepackage{microtype}      % para melhorias de justificação
\usepackage{float}

% --- Pacotes adicionais, usados apenas no âmbito do Modelo Canônico do abnteX2 ---
\usepackage{lipsum}        % para geração de dummy text

% --- Pacotes de citações ---
\usepackage[brazilian,hyperpageref]{backref} % Paginas com as citações na bibl
\usepackage[alf]{abntex2cite}                % Citações padrão ABNT

% --- CONFIGURAÇÕES DE PACOTES ---

% ---  Configurações do pacote backref usado sem a opção hyperpageref de backref
\renewcommand{\backrefpagesname}{Citado na(s) página(s):~} % Texto padrão antes do número das páginas
\renewcommand{\backref}{}

% Define os textos da citação
\renewcommand*{\backrefalt}[4]{
  \ifcase#1     Nenhuma citação no texto.  \or{}
    Citado na página #2.  \else
    Citado #1 vezes nas páginas #2.  \fi}% ---
\setlength{\parindent}{1.3cm} % --- Espaçamentos entre linhas e parágrafos --- %
\setlength{\parskip}{0.2cm}  % Controle do espaçamento entre um parágrafo e outro/tente também \onelineskip
% Seleciona o idioma do documento (conforme pacotes do babel)
%\selectlanguage{english}
\selectlanguage{brazil}
\frenchspacing % Retira espaço extra obsoleto entre as frases.

\makeindex % --- compila o índice ---

% ADIÇÕES FEITAS POR MIM (João Vítor Fernandes Dias)

\usepackage{booktabs} % Necessários pra tabela
\usepackage{multirow} % Necessários pra tabela
\usepackage{animate}  % Necessário para animações
\usepackage{listings} % Necessário para código
\usepackage{caption}  % Colocar caption em códigos na parte de cima
\usepackage{adjustbox} % Adicionando Easter Eggs
\usepackage{ulem} % Adicionado para cortar texto
\usepackage{attachfile} % Anexando código ao PDF: alternativas: embedfile, navigator
\usepackage[table, dvipsnames]{xcolor} % Adding color to table line. Não tenho certeza para que o dvipsnames serve
\usepackage{subcaption} % Adicionar subfiguras
\usepackage{chngcntr} % Para resetar o contador de figuras
\usepackage[final]{pdfpages} % Para incluir PDFs (folha de aprovação)

% \usepackage[english]{babel} % Para definir o Mês e Ano na folha de rosto
% \usepackage{datetime2} % Para definir o Mês e Ano na folha de rosto

% % Definindo o formato de data para mês e ano
% \DTMnewdatestyle{monthyear}{%
%   \renewcommand{\DTMdisplaydate}[4]{%
%     \DTMmonthname{##2}~##1%
%   }
%   \renewcommand{\DTMDisplaydate}{\DTMdisplaydate}
% }
% % Ativando o estilo de data definido
% \DTMsetdatestyle{monthyear}


\counterwithin{figure}{chapter}   % Resetar o contador de figuras a cada capítulo {chngcntr}
\counterwithin{table}{chapter}    % Resetar o contador de tabelas a cada capítulo {chngcntr}
% \counterwithin{lstlisting}{chapter} % Resetar o contador de tabelas a cada capítulo {chngcntr}

\attachfilesetup{
% color = {0 1 1}, % Cyan
color = {.05 .55 .18},
icon = Paperclip,
timezone = {-03'00'},
date={D:20240423173000000'00'},
mimetype = {application/x-rar-compressed},
% mimetype = {application/zip}
author = {Jo\~{a}o V\'{i}tor Fernandes Dias},
% author = {João Vítor Fernandes Dias},
description = {Anexos da minha monografia},
subject = {Anexos da minha monografia. Mais especificamente, o seu código fonte.},
}

% --- Configurações de aparência do PDF final: alterando o aspecto da cor azul
\definecolor{hyperBlue}{HTML}{2905C3}

% Cores das tabelas em 2.02!7-resultados
\definecolor{myCellColor}{HTML}{79b8ff}
\definecolor{myRemoveLineColor}{HTML}{b31d28}
\definecolor{myAddLineColor}{HTML}{22863a}
\definecolor{hyperBlue}{HTML}{2905C3}

% Cores das Listings
\definecolor{codegreen}{HTML}{009900}
\definecolor{codegray}{HTML}{808080}
\definecolor{codepurple}{HTML}{9400D3}
\definecolor{backcolour}{HTML}{F2F2EB}

% Colore tabelas em 2.02!7-resultados
\newcommand{\altered}{\cellcolor{myCellColor}} % Define a cor desta célula
\newcommand{\removeLine}{\rowcolor{myRemoveLineColor}} % Define a cor da próxima linha
\newcommand{\addLine}{\rowcolor{myAddLineColor}} % Define a cor da próxima linha

\renewcommand{\lstlistingname}{Código}% Listing -> Código
\renewcommand{\lstlistlistingname}{Lista de \lstlistingname s} % List of Listings -> Lista de códigos

% Necessário para códigos bonitos

\lstdefinestyle{mystyle}{
  backgroundcolor=\color{backcolour},
  commentstyle=\color{codegreen},
  keywordstyle=\color{magenta},
  numberstyle=\tiny\color{codegray},
  stringstyle=\color{codepurple},
  basicstyle=\ttfamily\footnotesize,
  breakatwhitespace=false,
  breaklines=true,
  captionpos=t,
  keepspaces=true,
  numbers=left,
  numbersep=5pt,
  showspaces=false,
  showstringspaces=false,
  showtabs=false,
  tabsize=2
}

% numberbychapter=⟨true|false⟩ true % If true, and \thechapter exists, listings are numbered by chapter. Otherwise, they are numbered sequentially from the beginning of the document. This key can only be used before \begin{document}.

\lstset{style=mystyle}

\def\selfAuthor{Fonte: autoria própria} % Fonte padrão para figuras

% Figuras automaticamente centralizadas com fonte
\newenvironment{CenteredFigure}{\begin{figure}[htbp]\centering}{\end{figure}}
\newenvironment{MyCenteredFigure}{\begin{CenteredFigure}}{\legend{\selfAuthor}\end{CenteredFigure}}

\newenvironment{CenteredTable}{\begin{table}[htbp]\centering}{\end{table}} % Centered Tables

% Adicionar comando mais denso para adição da animação
\newcommand{\myAnimation}[1]{\includegraphics[scale=0.2]{files/img/2.02!5-desenvolvimento/2.02!5.4-sistema/5.4.0-programacao/#1}}

% --- CONFIGURANDO LISTING JAVASCRIPT ---
\lstdefinelanguage{JavaScript}{
  keywords={typeof, new, true, false, catch, function, return, null, catch, switch, var, if, in, while, do, else, case, break},
  keywordstyle=\color{blue}\bfseries,
  ndkeywords={class, export, boolean, throw, implements, import, this},
  ndkeywordstyle=\color{darkgray}\bfseries,
  identifierstyle=\color{black},
  sensitive=false,
  comment=[l]{//},
  morecomment=[s]{/*}{*/},
  commentstyle=\color{purple}\ttfamily,
  stringstyle=\color{red}\ttfamily,
  morestring=[b]',
  morestring=[b]"
}
% --- CONFIGURANDO LISTING JAVASCRIPT ---

% Coloring YAML: https://tex.stackexchange.com/questions/152829/how-can-i-highlight-yaml-code-in-a-pretty-way-with-listings

% \definecolor{commentColor}{HTML}{54A668}
% \definecolor{keyWordColor}{HTML}{569CD6}
% \definecolor{textColor}{HTML}{CE9178}

% \newcommand\YAMLcolonstyle{\color{commentColor}\mdseries}
% \newcommand\YAMLkeystyle{\color{keyWordColor}\bfseries}
% \newcommand\YAMLvaluestyle{\color{textColor}\mdseries}

\newcommand{\YAMLcolonstyle}{\color{red}\mdseries}
\newcommand{\YAMLkeystyle}{\color{black}\bfseries}
\newcommand{\YAMLvaluestyle}{\color{blue}\mdseries}

\makeatletter
% here is a macro expanding to the name of the language
% (handy if you decide to change it further down the road)
\newcommand\language@yaml{yaml}

\expandafter\expandafter\expandafter\lstdefinelanguage
\expandafter{\language@yaml}
{
keywords={true,false,null,y,n},
keywordstyle=\color{darkgray}\bfseries,
basicstyle=\YAMLkeystyle,                                 % assuming a key comes first
sensitive=false,
comment=[l]{\#},
morecomment=[s]{/*}{*/},
commentstyle=\color{purple}\ttfamily,
stringstyle=\YAMLvaluestyle\ttfamily,
moredelim=[l][\color{orange}]{\&},
moredelim=[l][\color{magenta}]{*},
moredelim=**[il][\YAMLcolonstyle{:}\YAMLvaluestyle]{:},   % switch to value style at :
morestring=[b]',
morestring=[b]",
literate =    {---}{{\ProcessThreeDashes}}3
{>}{{\textcolor{red}\textgreater}}1
{|}{{\textcolor{red}\textbar}}1
{\ -\ }{{\mdseries\ -\ }}3,
}

% switch to key style at EOL
\lst@AddToHook{EveryLine}{\ifx\lst@language\language@yaml\YAMLkeystyle\fi}
\makeatother

\newcommand\ProcessThreeDashes{\llap{\color{cyan}\mdseries-{-}-}}

% \newcommand{\importTeX}[1]{\include{files/tex/#1}} % Importar arquivos tex de forma mais enxuta % contra: perde os hiperlinks do vscode

\newcommand{\LinkToURL}[2]{\href{#1}{#2}\footnote{\url{#1}}}

% --- COMANDOS PARA CRIAR AS QUESTÕES DO FORMULÁRIO ---

\newcommand{\MyCheckbox}[2] {
  \item \CheckBox[name=#2, checkboxsymbol=\ding{53}]{ } #1
}

\newcommand{\FormInNewLine}[1]{
  \begin{description}
    \item[] #1
  \end{description}
}

\newcommand{\QuestionNameOptions}[3]{
  \item #1
  \FormInNewLine{\ChoiceMenu[print, combo, name=#2]{ }{#3}}
}

\newcommand{\ChoiceMenuPeriodos}[2]{
  \QuestionNameOptions{#1}{#2}{ 0, 1, 2, 3, 4, 5, 6, 7, 8, 9, 10 }
}

\newcommand{\ChoiceMenuSNO}[2]{
  \QuestionNameOptions{#1}{#2}{1. Sim, 2. Não, 3. Outro }
}

\newcommand{\ChoiceMenuDdNcC}[2]{
  \QuestionNameOptions{#1}{#2}
  {
    1. Discordo completamente,
    2. Discordo parcialmente,
    3. Não tenho preferência,
    4. Concordo parcialmente,
    5. Concordo completamente
  }
}

% Seguindo a NBR 6023
\makeatletter
\@ifpackageloaded{url}{%
  \addtociteoptionlist{abnt-url-package=url}
  \def\UrlLeft{}
  \def\UrlRight{}
  \urlstyle{same}}
\makeatother
% Seguindo a NBR 6023

% \def\DataImpressao{\today} % Data atual
\def\DataImpressao{2 de julho de 2024} % Data de defesa

% --- DEFININDO LINKS USADOS AO LONGO DO DOCUMENTO ---

% 2.01!1-introducao
\def\LinkDrawio{https://www.drawio.com} % 2.02!3-organizacao; 2.02!4-modelagem
\def\LinkMermaid{https://mermaid.js.org}

% 2.02!2-marco
\def\LinkSistemaLECIV{https://uenf.br/cct/leciv/graduacao/horario-2017-01/}

% 2.02!3-organizacao
\def\LinkEstatuto{https://uenf.br/UENF_ARQUIVOS/Downloads/REITORIA_1360_1101117875.pdf}
\def\LinkSiteUENF{https://uenf.br/portal}
\def\LinkCCH{https://uenf.br/cch}
\def\LinkCCT{https://uenf.br/cct}
\def\LinkCBB{https://uenf.br/cbb}
\def\LinkCCTA{https://uenf.br/ccta}
\def\LinkLaboratorios{https://uenf.br/cct/administracao/laboratorios}
\def\LinkLAMET{https://uenf.br/cct/lamet}
\def\LinkLCFIS{https://uenf.br/cct/lcfis}
\def\LinkLECIV{https://uenf.br/cct/leciv}
\def\LinkLCQUI{https://uenf.br/cct/lcqui}
\def\LinkLAMAV{https://uenf.br/cct/lamav}
\def\LinkLCMAT{https://uenf.br/cct/lcmat}
\def\LinkLEPROD{https://uenf.br/cct/leprod}
\def\LinkLENEP{https://uenf.br/cct/lenep}
\def\LinkLicMat{https://uenf.br/posgraduacao/licenciatura-matematica}
\def\LinkSiteCCUENF{https://cc.uenf.br}
\def\LinkLCMATPósGraduação{https://uenf.br/posgraduacao/matematica/apresentacao}
\def\LinkUENFPósGraduações{https://uenf.br/posgraduacao/programas/pos-graduacao}
\def\LinkProfMat{https://profmat-sbm.org.br}
\def\LinkDistribuiçãoDeSalas{https://uenf.br/cct/secretaria-academica/distribuicao-das-salas-de-aula-do-cct}
\def\LinkRabbitMQ{https://rabbitmq.com}
\def\LinkGitLab{https://about.gitlab.com}

% 2.02!5-desenvolvimento
\def\LinkOurClass{https://jvfd3.github.io/timetabling-UENF}
\def\LinkPython{https://www.python.org}
\def\LinkProjetoDemanda{https://github.com/jvfd3/university_demand}
\def\LinkPDFMiner{https://pypi.org/project/pdfminer}
\def\LinkLGPD{https://www.planalto.gov.br/ccivil_03/_ato2015-2018/2018/lei/l13709.htm}
\def\LinkLGPDEstudoTécnico{https://www.gov.br/anpd/pt-br/assuntos/noticias/sei_00261-000810_2022_17.pdf}
\def\LinkFigma{https://www.figma.com}
\def\LinkVelcro{https://en.wikipedia.org/wiki/Hook-and-loop_fastener}
\def\LinkGitHubProjects{https://docs.github.com/pt/issues/planning-and-tracking-with-projects}
\def\LinkJSONBin{https://jsonbin.io}
\def\LinkMySQL{https://www.mysql.com}
\def\LinkAxios{https://axios-http.com}
\def\LinkExpress{https://expressjs.com}
\def\LinkNodeJS{https://nodejs.org}
\def\LinkReactRouter{https://reactrouter.com}
\def\LinkReactSelect{https://react-select.com}
\def\LinkBibliotecaGHPages{https://www.npmjs.com/package/gh-pages}
\def\LinkGitHubPages{https://pages.github.com}
\def\LinkAcademicoUENF{https://academico.uenf.br}

% 2.02!6-experimentos
\def\LinkLiveSplit{https://livesplit.org}
\def\LinkNotebook{https://www.asus.com/br/supportonly/x571gt/helpdesk_manual}

% 2.02!7-conclusão
\def\LinkIssues{https://github.com/jvfd3/timetabling-UENF/issues}
\def\LinkRepo{https://github.com/jvfd3/timetabling-UENF}
\def\LinkProjects{https://github.com/users/jvfd3/projects/3}

% 3.03#-apendice
\def\LinkCodigoFonteSistema{https://github.com/jvfd3/timetabling-UENF}
\def\LinkAdobeReader{https://get.adobe.com/br/reader}
\def\LinkCodigoFonteMonografia{https://github.com/UENF-Organizacao-de-Disciplinas/INF01131-Monografia}
\def\LinkDisciplinas{https://github.com/UENF-Organizacao-de-Disciplinas}
\def\LinkProjetoFigma{https://www.figma.com/design/a2CoAWN9V2IKHiZvtvX3AQ}
\def\LinkFormulario{https://forms.gle/zFfbvVtmKTVPX3cg9}

