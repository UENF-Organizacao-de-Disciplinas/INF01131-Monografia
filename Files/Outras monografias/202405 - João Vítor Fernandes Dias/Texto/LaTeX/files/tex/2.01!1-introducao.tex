\chapter{Introdução} \label{chap:introducao}             % ##   1

No ensino superior brasileiro, cada curso de uma instituição de ensino tem em seu projeto pedagógico, ou seja, no documento que rege quais as atribuições e justificativas de existência do curso, uma listagem de disciplinas a serem ministradas em cada semestre ao longo de sua duração esperada. Disciplinas estas que para serem cursadas os discentes precisam cumprir determinados requisitos.

Embora haja o planejamento de duração do curso, diversos fatores podem influenciar a previsão, dentre eles podemos citar eventos como: quebra de pré-requisitos, trancamento de matrícula, transferências, reprovações, indisponibilidade de professores, greves, dentre tantos outros.

Estes eventos tendem a, no geral, aumentar o tempo médio para conclusão do curso. Situação em sua maioria indesejada tanto pelos alunos, que mesmo durante seu estudo já visam o mercado de trabalho, quanto pelos professores e pela instituição, visto que a evasão do ensino superior brasileiro é um problema existente e estudado a fim de ser minimizado.

Com isso, é esperado que a instituição busque alternativas para tornar mais dinâmica e atrativa a experiência dos discentes durante sua jornada. Uma dessas formas é tentando minimizar o impacto que os atrasos na grade causam nos semestres consecutivos. Para isso sendo então necessária uma análise das disciplinas que devem ser oferecidas no próximo semestre, sendo então necessário definir \textbf{quais}, \textbf{quando}, \textbf{onde}, \textbf{por quem} e \textbf{para quem} serão ministradas. Esta tarefa, entretanto, não é trivial.

\section{O problema} \label{sec:Problemáticas}        % ###  1.1

Embora seja um problema atualmente, isso não significa que seja recente. Desde 1978 \cite{Barham1978} o termo \textit{timetabling} encontra-se no meio acadêmico como o termo referente ao construção de grade horária, sendo este o ramo de pesquisa que direciona este trabalho. Neste artigo de 1978 já se propunha uma forma para que se obtivesse um construção otimizada, e demonstrava que o método utilizado gerava bons resultados.

Outra característica deste problema é informada por \citeonline{Thomas2009} que fala sobre a multidimensionalidade do problema de \textit{timetabling}. Por causa dessa questão há uma complexidade elevada para conseguir conceber visual e mentalmente de que forma os dados relacionados ao problema se estruturam, assim dificultando a elaboração de sistemas computacionais que auxiliem nessa tarefa.

Também segundo \citeonline{Miranda2012}, embora o problema de atribuição de salas não seja novo e tenha extensa literatura a seu respeito, são poucos os que de fato implementaram um sistema para suporte de decisões. Isso se dá por diversos fatores, também listado pelo autor fazendo referência a trabalhos anteriores, sendo alguns deles a resistência organizacional a mudanças e adoção de novas tecnologias, nível de dificuldade do problema, dentre outros. Algumas outras características que se apresentam como problemas são a falta de otimalidade das grades horárias desenvolvidas em boa parte das instituições de ensino superior e a quantidade de tempo necessária para a criação dessas grades não-ótimas.

Em se tratando do caso específico da UENF, \textbf{o problema} que será abordado, consiste na criação de grades horárias se apresenta com suas próprias peculiaridades, dentre elas listam-se a aquisição de uma estimativa de alunos que desejam cursar determinada disciplina, a existência de diversas etapas na criação da grade que envolvem diferentes entidades da instituição cada uma delas contando com informações incompletas durante o processo, que vão aos poucos convergindo para uma solução final, que mesmo que não ótima, é uma solução viável. Ainda assim, a quantidade de tempo necessária para a criação de uma grade horária é extensa e não garante a otimalidade da solução encontrada.

Considerando que situações como a descrita acima são passíveis de ocorrer, e que a tarefa de criação de grades horárias é recorrente, um sistema de suporte à decisão que auxilie na centralização dos dados e que auxilie na percepção de alocações inapropriadas se faz necessário.

\section{Hipótese} \label{sec:Hipótese}                  % ###  1.2

Dada as características intrínsecas ao problema de agendamento de grade horária, é esperado que os \textit{softwares} atualmente existentes que lidam com este problema não apresentem completas capacidades de se moldar ao caso de uma instituição específica. E, caso a primeira hipótese se apresente correta, o \textit{software} a ser desenvolvido, assim como seus similares, se apresentará como uma solução plausível para a resolução do problema proposto embora ainda apresente melhorias possíveis a serem implementadas. O \textit{software} se apresentará de tal forma que cada uma das partes interessadas possam utilizá-lo de forma independente, sem se limitar à ausência de informações advindas das outras partes.
% Eu poderia adicionar aqui a hipótese de que a atual criação de grades horárias é falha e ineficiente ou algo mais suave.

\section{Objetivos} \label{sec:Objetivos}                % ###  1.3

Os objetivos desta monografia podem ser divididos entre gerais e específicos, não havendo relação de superioridade de um em relação ao outro, visto que ambos igualmente nortearão o desenvolvimento da pesquisa.

% \subsection{Gerais} \label{ssec:Gerais}                % #### 1.3.1

Como \textbf{objetivos gerais} temos o desenvolvimento de um sistema de suporte à decisão tal que aumente a eficiência, eficácia e efetividade do processo de criação de grades horárias que semestralmente demandam extensa quantidade de tempo dos coordenadores de curso na UENF e não alcançam a otimalidade. Em particular, nesse trabalho, trata-se do caso específico do curso da Ciência da Computação. Nesse processo, também é esperado que as grades horárias finais tragam a satisfação de todos os participantes do processo, desde os coordenadores de curso até os alunos.

% \subsection{Específicos} \label{ssec:EspeEspecíficos}  % #### 1.3.2

Como \textbf{objetivos mais específicos}, podemos listar os seguintes:

\begin{itemize}
  \item Entender de que forma os setores administrativos da UENF lidam com a questão do \textit{timetabling};
  \item Atender às principais necessidades dos responsáveis pela criação de grades horárias;
  \item Modelar o sistema de \textit{timetabling} de acordo com os requisitos demandados;
  \item Incentivar o uso do \textit{software} para a auxiliar na criação da grade horária.
\end{itemize}

\section{Justificativas} \label{sec:Justificativas}      % ###  1.4

% Levando em conta a problemática evidenciada e os sucessos prévios dos artigos anteriores, vê-se grande potencial de auxílio e aumento na satisfação de todos os que utilizarem os métodos propostos.
Não havendo um sistema geral que solucione todos os casos como evidenciado pelos pesquisadores da área, resta aos interessados rumarem em busca de uma solução entalhada nos moldes de sua instituição específica.

No caso específico na UENF, as grades horárias são construídas gradualmente e constantemente trabalham com informações incompletas, o que dificulta a percepção de alocações inapropriadas. Para tanto, sendo assim, faz-se válida a pesquisa e desenvolvimento de um sistema de suporte à decisão que seja capaz de lidar com a ausência de informações e a percepção dos conflitos advindos de alocações indevidas.

\section{Metodologia} \label{sec:Metodologia}            % ###  1.5

Considerando as dificuldades encontradas em trabalhos anteriores, entende-se que o maior desafio será superar as especificidades que serão encontradas durante a modelagem da universidade em questão. Para isso, será inicialmente necessária uma pesquisa bibliográfica com foco no estudo das abordagens qualitativas realizadas anteriormente que obtiveram sucesso em elicitar os requisitos adequados para as instituições de ensino.

Com este conhecimento, um material inicial para a pesquisa exploratória e qualitativa deve ser desenvolvido levando em conta as questões próprias da universidade em questão, visando também coletar dados relevantes para uma futura pesquisa com maior enfoque em características emergentes que a pesquisa anterior pode levantar, similar a como foi proposto e realizado por \citeonline{Andre2018}. Esta pesquisa exploratória sobre a realidade da instituição se subdivide em três frentes: o estudo sobre a \textbf{documentação teórica}, \textbf{entrevista} com os \textit{stakeholders} (gestores envolvidos na elaboração do quadro de horários) e \textbf{formulário} direcionado aos alunos.

% Na \textbf{documentação teórica}, espera-se encontrar informações sobre a estrutura organizacional da UENF, bem como as regras que regem a criação de grades horárias. Quais são os cargos envolvidos e quais são as responsabilidades de cada um deles. Com esta informação, estabelecem-se assim os \textit{stakeholders} iniciais. Na \textbf{entrevista}, algumas informações esperadas giram em torno das percepções dos \textit{stakeholders} do sistema proposto. Estas percepções incluem o entendimento deles quanto ao método atual e às alternativas existentes, nível de insatisfação com o método atual, nível de desejo quanto à um novo método. Além disso, espera-se aproveitar o ensejo para elicitar as características e funcionalidades que gostariam de ter em um sistema de suporte à decisão, solicitando também que deem informações adicionais que gostariam de acrescentar. \textbf{Questionamentos} também serão realizados com alunos, porém em formato de formulário online para atingir mais objetivamente uma quantidade mais vasta de respondentes. Espera-se obter informações sobre a satisfação dos alunos com as grades horárias atuais, para assim poder confirmar se a insatisfação é generalizada e percebida por todos os envolvidos.

Sendo compreendido então o cenário atual da universidade, será então necessário modelar o sistema de suporte à decisão de acordo com as especificidades encontradas, compilando, de maneira geral, os estágios do Design de Interação, o funcionamento do sistema, o diagrama conceitual do seu banco de dados, e suas principais diferenças em relação aos trabalhos anteriores.

Por fim, será apresentado o processo do desenvolvimento do sistema, partindo dos projetos anteriores que motivaram a execução deste trabalho, a tentativa de acesso aos dados acadêmicos da instituição, a prototipação do sistema, e o desenvolvimento do sistema em si, que foi subdividido em três diferentes versões, encerrando com mais detalhes sobre as duas características mais notáveis deste sistema: a resolução dos conflitos percebidos e a criação automatizada de uma grade horária inicial.

% DAR MAIS ENFOQUE A TUDO O QUE FOI USADO

\section{Organização} \label{sec:Organização}            % ###  1.6

Este trabalho abordará capítulos que tratam dos seguintes tópicos:

\begin{itemize}
  % \item O \autoref{chap:introducao} de introdução traça informações gerais sobre o assunto do trabalho, elaborando mais detalhadamente quanto à sua \hyperref[sec:Problemáticas]{problemática}, \hyperref[sec:Hipótese]{hipótese}, \hyperref[sec:Objetivos]{objetivos}, \hyperref[sec:Justificativas]{justificativas}, a \hyperref[sec:Metodologia]{metodologia escolhida} e a \hyperref[sec:Organização]{organização de suas informações};
  \item O \autoref{chap:marco} de revisão literária informa mais detalhadamente sobre os problemas de agendamento, suas categorias, soluções, desafios e definições de termos;
  \item O \autoref{chap:modelagem} de modelagem, apresenta-se a conceitualização macro de como o sistema deve se comportar, quais são as suas funcionalidades e quais são os seus objetivos;
  \item O \autoref{chap:instituicao} de contexto da instituição apresenta a \hyperref[sec:estatuto]{Universidade Estadual do Norte Fluminense Darcy Ribeiro (UENF), suas características, estrutura organizacional}, \hyperref[sec:entrevistas]{entrevistas com os \textit{stakeholders} relacionados à criação de grades horárias} e a \hyperref[sec:sequencia]{sequência de passos para a criação de uma grade horária};
  \item O \autoref{chap:desenvolvimento} de desenvolvimento, pode ser visto o processo de criação do sistema, contextualizado a partir dos projetos anteriores, passando pela prototipação, para então chegar ao desenvolvimento em três etapas do sistema final;
        % \item O \autoref{chap:experimentos} de experimentos apresenta a validação do sistema desenvolvido através de testes de criação de grade horária seguido de resolução dos conflitos encontrados;
  \item O \autoref{chap:resultados} de resultados demonstra as páginas do sistema desenvolvido, as funcionalidades implementadas e outras ferramentas disponíveis para a gestão dos horários;
  \item O \autoref{chap:conclusoes} finaliza o presente trabalho apresentando as conclusões indicando caminhos a serem seguidos por pesquisadores posteriormente.
        % \item O \autoref{chap:apendice} de apêndices apresenta informações adicionais que não se encaixam nos capítulos anteriores, mas que são relevantes para o entendimento do trabalho, sendo eles o código-fonte do sistema e da elaboração desse texto, a pesquisa realizada e a análise de seus resultados.
\end{itemize}
