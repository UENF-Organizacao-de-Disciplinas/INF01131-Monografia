% resumo em português
\setlength{\absparsep}{18pt} % ajusta o espaçamento dos parágrafos do resumo


\begin{resumo}
Esta monografia apresenta um estudo detalhado sobre a segurança em aplicações bancárias no sistema operacional Android, com o objetivo de desenvolver um modelo de requisitos de segurança específico para essas aplicações. Dada a crescente popularidade dos dispositivos móveis e a frequência com que são utilizados para realizar transações financeiras, garantir a segurança dessas operações tornou-se uma prioridade crucial. O trabalho realiza uma análise abrangente das vulnerabilidades e ameaças associadas aos aplicativos bancários no sistema operacional Android, além de explorar os métodos de mitigação mais eficazes para esses problemas. Para identificar e entender os pontos críticos de segurança, foi feita uma revisão na literatura e utilizada a ferramenta de análise de casos de uso indevido, permitindo uma compreensão profunda das possíveis falhas e riscos. A partir dessa análise, é proposto um modelo detalhado de requisitos de segurança, específico para o contexto das aplicações bancárias no Android, e estruturado através de diagramas UML. Este modelo inclui diretrizes claras e práticas para garantir a confidencialidade, integridade, autenticação e autorização dos dados. Finalmente, o modelo é avaliado por meio de um estudo de caso aplicado à Herd Financial e de uma avaliação de conformidade utilizando a certificação PCI DSS, demonstrando como os requisitos de segurança propostos podem ser implementados e como eles garantem a segurança da aplicação. Esta monografia modela uma solução prática e implementável para garantir a proteção dos dados dos usuários e a confiança nos serviços bancários móveis; assim, destacando a importância de medidas de segurança robustas e oferece uma contribuição significativa aos desenvolvedores e provedores desse serviço.

 \textbf{Palavras-chave}: Segurança da informação. Aplicações bancárias móveis. Android. Engenharia de requisitos.
\end{resumo}

% resumo em inglês
\begin{resumo}[Abstract]
 \begin{otherlanguage*}{english}
This monograph presents a detailed study on the security of banking applications on the Android operating system, with the goal of developing a specific set of security requirements for these applications. Given the growing popularity of mobile devices and their frequent use for financial transactions, ensuring the security of these operations has become a critical priority. The work provides a comprehensive analysis of the vulnerabilities and threats associated with banking applications on the Android operating system, as well as exploring the most effective mitigation methods for these issues. To identify and understand the critical security points, a literature review was conducted, and the misuse case analysis tool was used, allowing a deep understanding of potential failures and risks. Based on this analysis, a detailed security requirements model is proposed, specifically tailored to the context of banking applications on Android, structured through UML diagrams. This model includes clear and practical guidelines to ensure data confidentiality, integrity, authentication, and authorization. Finally, the model is evaluated through a case study applied to Herd Financial and an assessment of compliance with the PCI DSS certification, demonstrating how the proposed security requirements can be implemented and how they ensure the application's security. This thesis models a practical and implementable solution to ensure the protection of user data and trust in mobile banking services, highlighting the importance of robust security measures and offering a significant contribution to developers and providers of this service.
   \vspace{\onelineskip}
 
   \noindent 
   \textbf{Keywords}: Information security. Mobile Banking. Android. Requirements engineering.
 \end{otherlanguage*}
\end{resumo}