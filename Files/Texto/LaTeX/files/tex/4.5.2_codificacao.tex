\section{Modelo de Banco de Dados} % ### 5.4. Modelo de Banco de Dados

Considerando as informações necessárias para o presente trabalho, e também o preparo de campo para potenciais aplicações futuras, foi elaborado um diagrama conceitual de banco de dados, que pode ser visto na Figura \ref{fig:DiagramConceitual}.

\begin{figure}[htbp]\centering
  \caption{\label{fig:DiagramConceitual} Diagrama Conceitual do banco de dados}
  \includegraphics[scale=0.2]{files/img/DiagramaConceitual/DiagramaConceitualBranco.png}
  \legend{Fonte: o autor}
\end{figure} % Diagrama Conceitual

O diagrama conceitual foi elaborado utilizando a ferramenta \href{https://www.drawio.com/}{draw.io} citada na metodologia e ilustra as relações entre diversas entidades presentes na realidade da UENF. O emaranhamento presente no diagrama ilustra a complexidade envolvida na criação de uma grade horária, onde diversas entidades se relacionam entre si.

Como principais apontamentos, podemos citar a parte principal do modelo que é a alocação de turmas. Ela, como já descrito, envolve a correlação entre alunos de diferentes cursos, professores, disciplinas, salas e horários. Além disso, também é possível notar a presença de entidades que não são diretamente relacionadas à alocação de turmas, mas que podem se mostrar úteis, como a relação entre professores e laboratórios, e a de disciplinas e ementas.

Embora o diagrama apresente uma visão mais completa de todas as interconexões possíveis, é importante ressaltar que o presente trabalho foca primordialmente na alocação das turmas para o curso de Ciência da Computação, e que a implementação do banco de dados será feita de forma a atender a essas necessidades, fazendo então uso de uma parte do diagrama conceitual.

% Tendo isso em vista, o modelo conceitual reduzido para o presente trabalho pode ser visto na Figura \ref{fig:DiagramConceitualReduzido}.

% \begin{figure}[htbp]\centering
%   \caption{\label{fig:DiagramConceitualReduzido} Diagrama Conceitual reduzido para o presente trabalho}
%   \includegraphics[scale=0.2]{files/img/DiagramaConceitual/DiagramaConceitualReduzidoBranco.png}
%   \legend{Fonte: o autor}
% \end{figure} % Diagrama Conceitual reduzido

Neste modelo, mais enxuto, temos apenas as entidades principais, onde temos uma turma de determinada disciplina, ministrada por um professor e que ocorre em uma sala em um determinado horário.

\subsection{Diagrama de Entidade e relacionamento} % ### 5.4.1. Modelo Relacional

% Com isso, restamos o diagrama de Entidade e Relacionamento (DER) que pode ser visto na Figura \ref{fig:DER}.

% \begin{figure}[htbp]\centering
%   \caption{\label{fig:DER} Diagrama de Entidade e Relacionamento}
%   \includegraphics[scale=0.2]{files/img/DER/DERBranco.png}
%   \legend{Fonte: o autor}
% \end{figure} % Diagrama de Entidade e Relacionamento

Neste diagrama vemos as entidades principais, que são \textit{Turmas}, \textit{Disciplinas}, \textit{Professores}, \textit{Horários} e \textit{Salas}. As propriedades escolhidas para cada entidade são compostas por uma mistura de critérios. Por exemplo, o nome do professor, o código da disciplina, e a junção de código e bloco auxiliam primordialmente na identificação real dos professores, disciplinas e salas. Já as informações ``período'', ``apelido'' e ``comment''...

E também é notável a presença da entidade \textit{Alunos}, que se apresenta desacoplado das demais entidades. O motivo para isso é que, embora os alunos façam parte do processo de alocação de turmas, ao longo do desenvolvimento, o desenvolvimento de funcionalidades envolvendo os alunos...

\section{História do desenvolvimento} % ### 5.5.1. História do desenvolvimento

%% Acho que vai ficar mais fácil assim

Após a elaboração dos protótipos com o Figma, e da conceitualização diagramática do banco de dados, o desenvolvimento do sistema foi iniciado. Por maior familiaridade com a linguagem e considerando que é uma das mais utilizadas no mercado para desenvolvimento web [buscar referência], foi escolhida a linguagem JavaScript, utilizando a biblioteca React para a criação dos componentes visuais, ou seja, o front-end, e o Node.js para a criação do back-end e a criação de um servidor local que permitiria visualizar as mudanças no código em tempo real.

% Perguntar pra quem sabe, sobre qual é o real papel do Node.js

\subsection{Primeira versão} % ### 5.5.2. Primeira versão

A primeira versão do sistema foi desenvolvida em um ambiente local, com o objetivo de se aproximar ao máximo das páginas previstas no protótipo. Para isso, foi utilizada a biblioteca React Router para a navegação entre as páginas, e a biblioteca React Select para as caixas de seleção.

\subsubsection{Banco de Dados primitivo}

Os dados contidos no sistema foram inicialmente armazenados em arquivos JSON, que eram importados diretamente para o código. Isso foi feito para que fosse possível visualizar o funcionamento do sistema sem a necessidade de um banco de dados real. A partir disso, foi possível visualizar o funcionamento do sistema e realizar testes de usabilidade. Em contrapartida, os dados disponíveis não eram modificáveis, tendo apenas a possibilidade de leitura e mutação temporária, visto que após recarregar ou mudar de página, as mudanças eram perdidas.

Nesse método, cada entidade era armazenada em um arquivo JSON separado, contendo esse um array de objetos, onde em cada objeto haviam as chaves, representando as propriedades da entidade, e os valores, representando os dados da entidade.

Como nesta dinâmica não havia uma forte correlação entre os dados, o frontend acabava sendo o responsável por unir todas as informações. Assim, por exemplo, para se obter a lista de professores de uma turma, era necessário importar todos os professores, todas as turmas, e então, a partir do nome do professor alocado àquela turma, buscar na listagem dos professores qual era o professor que correspondia àquele nome, para então agregar as informações.

\subsubsection{Funcionalidades iniciais}

Nessa primeira versão, algumas funcionalidades já começaram a ser esboçadas, principalmente as funcionalidades CRUD (Create, Read, Update, Delete) para as entidades principais do sistema. Embora, como já dito, os dados não fossem persistentes, foi possível visualizar o funcionamento das funcionalidades de criação e leitura de turmas, professores, disciplinas, salas e horários.

Nessa versão, também foi implementada uma checagem bruta de conflitos por alocação simultânea de professores em mais de uma turma e a checagem da quantidade de demanda de alunos em relação à capacidade das salas. Uma descrição mais detalhada das funcionalidades de conflitos está presente \hyperref[Conflitos]{adiante}.

Além dessas funcionalidades que se mantiveram até a conclusão do sistema, também foram desenvolvidas funcionalidades que não obtiveram o mesmo êxito e que foram deixadas de lado ao decorrer do caminhar. Dentre elas, podemos citar a definição de níveis de preferência de horários para professores, a definição das características especiais das salas, e o andamento dos alunos em relação às disciplinas. Houveram também outras que nem chegaram a ser desenvolvidas, como a realocação de turmas através de um sistema de arrastar e soltar e o uso de heurísticas para a realocação de turmas.

\subsubsection{GitHub Pages}

Após o desenvolvimento local, como forma de viabilizar o acesso ao sistema por parte de outros usuários, foi feito o \textit{deploy}, ou seja, foi feito o upload do sistema para um servidor online. Para isso, foi utilizado o serviço GitHub Pages que, por ser gratuito e de fácil utilização, foi a escolha mais adequada para o momento. O sistema pode ser acessado através do link \url{https://jvfd3.github.io/timetabling-UENF/}.

\subsection{Segunda versão} % ### 5.5.3. Segunda versão

Utilizando do feedback quanto aos resultados entregues na primeira versão, alguns pontos de melhoria foram identificados, sendo um deles, e o mais importante: o planejamento. Na primeira abordagem, o desenvolvimento foi feito seguindo notas e ideias soltas, sem um planejamento prévio, o que resultou em um sistema que, embora funcional, não atendia a todas as necessidades propostas. E ia além: exibia funcionalidades que não eram de todo necessárias, ou melhor dizendo, têm menor prioridade do que muitas outras.

% Mesmo com esta nova dinâmica, outras funcionalidades foram deixadas de lado. Uma das que foram deixadas de lado foi a possibilidade de fixar certas informações. A proposta era que, certas disciplinas como Cálculo e Álgebra que são ofertadas para múltiplos cursos, pudessem ser fixadas em horários específicos, para que simplificasse aos coordenadores dos cursos a alocação de turmas.

\subsubsection{GitHub Projects}

Com isso, utilizando o GitHub Projects, foi organizado uma tabela de tarefas, onde foram unificadas as diversas anotações e ideias, antes soltas. A partir disso, foi possível visualizar o que era mais importante e o que poderia ser deixado de lado.

% [ADICIONAR IMAGEM DA TABELA BONITINHA: https://github.com/users/jvfd3/projects/3]

Tendo este novo sistema de tarefas em prática, foi possível tranquilizar a mente quanto ao conflito entre as funcionalidades que precisavam ser desenvolvidas, as que já estavam prontas, as que poderiam ter melhorias e quais se desejava implementar no futuro.

As tarefas foram inicialmente divididas em três principais categorias: \textit{Status}, \textit{Pages} e \textit{Sequence}. O \textit{Status} reflete o andamento da tarefa, se ela está disponível, em andamento, ou concluída. O \textit{Pages} reflete em qual página do sistema a tarefa se encontra, e o \textit{Sequence} reflete a ordem de prioridade da tarefa.

\subsubsection{Permanência dos dados}

Tendo agora uma rota mais clara a ser seguida, o desenvolvimento foi retomado. Uma das características mais marcantes e ainda não atribuídas ao sistema era a manutenção dos dados. Para exemplificar o funcionamento geral da permanência dos dados, consideremos o uso de uma \textit{REST API} utilizando de 4 ``camadas'': A. o frontend; B. os endpoints; C. as funções de execução; D. o banco de dados.

% [Usar diagrama para ilustrar]

O \textbf{frontend} é a interface do sistema, onde o usuário interage com o sistema. Ele se encontra em duas formas: a primeira é o chamado ``em produção'', que é o sistema que o usuário final acessa, e a segunda é o chamado ``em desenvolvimento'', que é o sistema que o desenvolvedor acessa para realizar as modificações necessárias. Ambas precisam se comunicar com o backend para realizar as quatro operações básicas no banco de dados (criação, leitura, atualização e deleção) por sobre as entidades existentes (turmas, professores, disciplinas, salas, etc.). Elas assim o fazer ao enviar requisições HTTP (GET, POST, PUT e DELETE), contendo pacotes de informações em formato JSON para os \textbf{endpoints}.

Os \textbf{endpoints} são as rotas que o backend disponibiliza para a recepção das requisições HTTP. Eles são responsáveis por encaminhar as requisições recebidas. Se funcionamento é simples: rotear as requisições recebidas junto com sua carga útil. Para tanto, as rotas criadas refletem diretamente a qual entidade do banco de dados a requisição se refere, sendo então assim sabido qual \textbf{função} deve ser executada.

As \textbf{funções} são as responsáveis por executar as operações no banco de dados. Elas processam o pacote de informações recebido, e então realizam a operação desejada no \textbf{banco de dados}.

O \textbf{banco de dados} recebe a requisição, processa a operação, e então retorna o status da operação. Esse retorno é então repassado camada por camada, até chegar ao frontend, onde o usuário final pode visualizar o resultado da operação.

\textbf{Resumidamente}: O \textbf{frontend} envia uma requisição HTTP com uma carga de informações a um \textbf{endpoint}, que encaminha a requisição a uma \textbf{função} específica que executa uma operação no \textbf{banco de dados}, assim retornando o status da operação ao frontend.

\paragraph*{\href{https://jsonbin.io/}{JSONBin}}

Como até então os dados estavam armazenados em formato JSON, imaginou-se que a melhor forma de persistir os dados seria através de um banco de dados que lidasse com JSON, e o escolhido foi o JSONBin.

Esta plataforma permite a criação de \textit{bins}, que são basicamente coleções de dados em formato JSON. A partir disso, é possível realizar requisições HTTP para a leitura, escrita, atualização e remoção dos dados. A utilização do JSONBin foi feita através de requisições HTTP usando o objeto \textit{XMLHttpRequest} do JavaScript, e a comunicação entre o frontend e o JSONBin foi feita através de \textit{tokens} de acesso.

Com isso, se tornou possível ler e atualizar os dados de forma remota, e assim, manter os dados mesmo após a recarga da página. Embora cumprisse com o que promete e o que era desejado, o JSONBin não se mostrou a melhor escolha para o sistema, visto que a sua utilização não performou tão bem quanto se esperava. Não se sabe se foi por inexperiência ou por limitações do próprio serviço, mas a utilização do JSONBin para a coleta dos dados, fazia com que a tela de carregamento do sistema demorasse alguns segundos para ser exibida, o que não é apropriado para a usabilidade do sistema proposto.


\section{Conflitos}\label{Conflitos} % ### 5.6. Conflitos

Uma das principais funcionalidades do sistema é a detecção de conflitos. Seu objetivo é auxiliar ao usuário a identificar possíveis problemas na alocação das turmas, e assim, permitir que ele possa corrigí-los antes de finalizar a grade horária. Diversas situações podem ser consideradas conflitos, e cada uma delas é tratada de forma diferente.

Deve-se ressaltar que o que aqui são chamados de conflitos, não são necessariamente o que são chamados de conflitos no contexto de programação linear. Aqui, conflitos são situações que podem ser consideradas inadequadas para a alocação de turmas, mas que ainda assim, não são completamente impeditivas para a alocação prática das turmas. Essa característica deve ser levada em consideração pois, apesar de geralmente retratar situações atípicas, é surpreendentemente recorrente a ocorrência de situações atípicas diversas.

\subsection{Típicos conflitos atípicos}

Para ilustração, abaixo estão descritos alguns exemplos de conflitos que poderiam ser alertadas pelo sistema, mas que não seriam realmente um restritor para a execução prática das alocações:

Considerando o diminuto corpo docente do curso de Ciência da Computação, que atualmente conta com seis professores doutores, é recorrente a solicitação de professores bolsistas para ministrar disciplinas. Devido aos prazos existentes ao longo do processo de criação da grade horária, é comum que ainda não se saibam quais e quantos professores bolsistas serão disponibilizados para quais turmas. Porém, como o Sistema Acadêmico requere a inserção de professores para a criação de turmas, uma solução encontrada foi a inserção de um desses professores permanentes como responsável pela turma. E, mesmo após se obter a informação quanto a quais e quantos bolsistas estarão disponíveis, ainda assim o sistema acadêmico não os permite serem inseridos, visto que eles não têm um vínculo permanente com a instituição. Com isso, seria possível ver, por exemplo, um conflito entre duas turmas que possuem o mesmo professor em um mesmo horário, mas que na prática, uma delas será ministrada por um professor bolsista.

Outras situações que podem ocorrer revolvem em torno da alocação das salas. Duas situações que podem ilustrar sua atipicidade são: a possibilidade de alocar uma turma a uma determinada sala, mesmo que se tenha a intenção de ministrá-la em outra, e também a possibilidade de se repartir a turma em duas salas de aula ocorrendo simultaneamente.

Esses e outros são exemplos de situações recorrentes ao longo do processo flexível da organização da tabela horária.

\subsection{Conflitos tratados pelo sistema}

Para a implementação, primeiro visou-se a detecção de conflitos que poderiam ser considerados restritores para a alocação das turmas. Sendo eles os de alocação simultânea de salas e professores, visto que um professor não pode ministrar duas turmas simultaneamente, nem uma sala deve comportar duas turmas simultaneamente (embora este segundo seja teoricamente possível).

Além disso, também foi implementada a detecção de conflitos de capacidade, onde a quantidade de alunos de uma turma é maior do que a capacidade da sala alocada, e alguns outros indicativos visuais que serão descritos abaixo.

Os conflitos calculados são representados de três formas diferentes. A primeira e mais perceptível é a mudança de cor de fundo das propriedades conflituosas. A segunda, visando evitar sobreposição de conflitos, é a adição de uma borda inferior que se estende por toda a largura da propriedade. E a terceira, mais descritiva, é o uso do atributo \textit{title} dos elementos HTML, que exibe uma mensagem de alerta flutuante ao passar o mouse sobre a propriedade conflituosa, assim dispondo de mais detalhes sobre os conflitos buscados e encontrados.

Embora o sistema seja projetado para ser permissivo quanto a inexistência de certas informações, é sempre esperado que a maior quantidade de informações possíveis seja inserida, assim, caso algum campo não tenha sido preenchido a cor de fundo do elemento será alterada para um tom acinzentado.

% "propriedades conflituosas" é um bom termo?

\subsubsection{Professores}

O sistema contempla a checagem de conflitos de alocação simultânea de professores em mais de uma turma. Ou seja, considerando todas as turmas ao qual o professor está atribuído no ano e semestre selecionados, o sistema compara todos os horários das turmas deste professor, e verifica se há alguma interseção entre horários que estão no mesmo dia, levando em conta a duração da aula.

Caso haja algum conflito, o sistema destaca o professor em questão, tornando a sua cor de fundo avermelhada. Além disso, ao passar o mouse sobre o nome do professor, é exibido um alerta flutuante, informando que quais são as turmas e horários que estão em conflito.

\subsubsection{Salas}

As salas também apresentam a verificação do conflito de alocação simultânea. Porém, diferente dos professores, a checagem é feita conferindo todos os horários na qual a sala está alocada, e então é feita a mesma verificação de interseção citada anteriormente. Havendo o conflito, é exibida uma borda alaranjada na parte inferior das propriedades referentes ao conflito, além de, assim como no caso dos professores, exibir o alerta flutuante.

Além disso, também é feita a comparação entre a quantidade máxima de alunos comportados na sala e a quantidade de alunos estimados para a turma. Este conflito por sua vez é ilustrado tornando avermelhado o fundo da demanda estimada e da seleção de salas. Caso uma turma tenha mais de um horário, é calculada a quantidade remanescente dos alunos que demandam a disciplina com relação a cada uma das capacidades das salas destes horários, mostrando cada um deles no alerta flutuante.

\subsubsection{Disciplina}

Além desses conflitos, outras características analisadas e representadas é quanto às disciplinas atribuídas às turmas, que, embora não representem necessariamente um \textit{conflito}, mas sim um indicativo, ainda assim serão tratatados como conflitos por motivos de simplificação. Esse indicativo leva em consideração o semestre selecionado e o período esperado da disciplina de certa turma. Utilizando de lógica similar, também é indicado caso não tenha sido atribuído um período à disciplina, e se, para o curso de Ciência da Computação, a disciplina é considerada como \textbf{Eletiva Livre}, \textbf{Eletiva Optativa}, ambas em tons azulados, ou se não é uma disciplina para o curso de Ciência da Computação, sendo então representada em tons alaranjados.

Os semestres possíveis são três: o primeiro semestre, o segundo semestre e o ``período de verão''. No caso do período de verão, as disciplinas que têm o seu período esperado neste semestre são marcadas com um tom amarelado, visto que não há relevância da sua paridade em um período de férias. Já nos casos das disciplinas de paridade ímpar (disciplinas dos períodos 1, 3, 5, 7 e 9) no primeiro semestre, ou as disciplinas de paridade par (disciplinas dos períodos 2, 4, 6, 8 e 10) no segundo semestre, estas são marcadas com um tom esverdeado, sendo aquelas referentes aos períodos finais do curso marcadas com um tom mais escuro. Já as disciplinas pares em semestres ímpares, ou as disciplinas ímpares em semestre pares, são ilustradas com a cor avermelhada, seguindo a mesma lógica de gradiente escuro nos últimos períodos.
