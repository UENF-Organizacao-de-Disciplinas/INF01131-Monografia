\chapter{Resultados} \label{chap:resultados} % ### 7.

Estando no início do fim, seguiremos agora numa jornada retroativa ao que foi feito até então e os resultados obtidos.

\section{Contexto Acadêmico} % ### 7.1.

Pudemos ver através da pesquisa acadêmica aos artigos e trabalhos relacionados com a área da construção de grades horárias em específico, o \textit{university course timetabling}, que a área é vasta, tanto em quantidade de artigos publicados quanto com muitas possibilidades de pesquisa e desenvolvimento que pode ser visto na revisão literária realizada por \citeonline{Alencar2019}. Uma das principais questões que mantêm a área em constante inconclusão é o nível de especificidade que cada instituição de ensino possui, o desafio deixa de ser a implementação do método resolutivo, e passa a ser modelar o problema específico e lidar com as preocupações dos usuários, como é concluído por \citeonline{Murray2007}.


A UENF não é diferente, já tendo sido alvo de pesquisas e desenvolvimentos de sistemas de otimização de grade horária em anos anteriores, em especial as monografias de \citeonline{Sanya2013} e \citeonline{Ricardo2014}, mas que não foram adotados pela instituição mesmo que representassem a eficiência da resolução do problema.

Uma alternativa encontrada para contornar a dificuldade encontrada pelos trabalhos anteriores na UENF no campo da modelagem do problema foi utilizar de uma abordagem voltada para a Interação Humano-Computador, como feito por \citeonline{Andre2018}, para a construção de um sistema de otimização de grade horária, que permitisse a participação ativa dos usuários na construção da grade horária, o que poderia facilitar a aceitação do sistema pela instituição.

\section{Estrutura da Instituição}

Visando enfrentar diretamente o problema da especificidade e modelagem do problema na UENF. Para tanto foi feito um estudo sobre a forma como os diversos setores relacionados à construção da grade horária interagem entre si, quais são os seus responsáveis e qual é a sequência de ações que cada setor realiza para a construção da grade horária.

O primeiro passo foi a leitura do

\section{Sistema}

O código fonte para o sistema desenvolvido está disponvível no \autoref{apendice:CodigoFonte}.

\subsection{Solução Ótima} % ### 7.2.

\subsection{Preparo para trabalhos futuros}

Ao longo de todo o desenvolvimento

\begin{MyCenteredFigure}
  \caption{Banco de Dados Final}
  \label{fig:BD_Final}
  % \includegraphics[width=\textwidth]{}
\end{MyCenteredFigure}

