% resumo em inglês
\begin{resumo}[Abstract]
  \begin{otherlanguage*}{english}

    This monograph presents a study on the creation of timetables in universities, focusing on the Universidade Estadual do Norte Fluminense, and the development of a decision support system for the creation of timetables. This process is complex and involves several factors, such as the availability of classrooms and teachers, the number of students and the demand for disciplines. As one of the greatest difficulties in the academic field of creating timetables is the modeling of the problem, this work proposes to structure the information regarding the institution studied through research and interviews. From this, it was possible to identify the sequence of creating the timetables and the restrictions imposed by the institution, thus being able to propose a system that assists the process. For this, the system was developed using JavaScript and the React library, being hosted on GitHub Pages and using Amazon Web Services for the execution of its backend and data storage. Its features include the CRUD (create, read, update and delete) functionalities of teachers, disciplines, classrooms, students, classes and schedules, the automated creation of an initial timetable for the Computer Science course at UENF, the visualization of conflicts among the allocated resources and the historical analysis of the data stored in the system.

    \textbf{Keywords}: Timetabling. University Class Scheduling. Heuristics. Interaction Design. Human Computer Interaction.

  \end{otherlanguage*}

  % \vspace{\onelineskip}
  % \noindent

\end{resumo}