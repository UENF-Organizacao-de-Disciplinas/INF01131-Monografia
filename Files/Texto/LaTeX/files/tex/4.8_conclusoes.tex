\chapter{Conclusões} % ## 8. Conclusões

O problema de organização de grade horária no ensino superior tem sido amplamente estudado por diversos pesquisadores. Devido sua natureza multidimensional e com forte tendência a especificidades, este campo de estudo se mostra como amplo e complexo.

Através da revisão bibliográfica, foi possível observar que a maioria dos trabalhos se foca em um método heurístico de solução, onde se busca uma solução ótima, ou próxima do ótimo, através de um método de busca. Entretanto, o presente trabalho se propõe a uma abordagem diferente, onde se busca uma solução boa o suficiente para que seja utilizada na prática, mesmo que não seja ótima, isso através do método de manipulação manual dos dados.

Para este fim foi desenvolvido um sistema de suporte à decisão para auxiliar os setores da Universidade Estadual do Norte Fluminense Darcy Ribeiro (UENF) responsáveis pela criação de grades horárias. O sistema foi desenvolvido com o intuito de ser utilizado como uma ferramenta auxiliar, onde os usuários possam manipular os dados de forma mais intuitiva e visual, assim reduzindo a necessidade de retrabalho e aumentando a produtividade.

O sistema permite que as quatro operações básicas de armazenamento persistente, sendo elas a criação, leitura, edição e deleção de dados. Com isso, os usuários podem adicionar manualmente as informações referentes ao trabalho de criação de grade horária de forma centralizada, assim reduzindo a necessidade de se lidar com diversos arquivos e planilhas. Facilitando também a visualização de informações, como a alocação de turmas, que pode ser visualizada de forma gráfica, assim facilitando a identificação de conflitos e problemas. O que consequentemente tende a agilizar o processo de busca por novas soluções e a redução dos conflitos.

% <!-- Tendo atingido este objetivo, espera-se que o sistema possa ser utilizado na prática, e que possa ser aprimorado e melhorado com o tempo. -->

\section{Trabalhos futuros} % ## 8.1 Trabalhos futuros

Como trabalhos futuros, vê-se uma ampla gama de pesquisa e aprimoramento ao presente trabalho, visto que este busca um método alternativo de solução ao mesmo problema abordado por outros dois pesquisadores em tempos anteriores. Pode-se então elaborar uma conexão entre o atual sistema e modelos aos métodos heurísticos propostos, permitindo então uma abordagem híbrida humano-computador na busca da grade horária ótima. Sugere-se inclusive o estudo sobre a aplicação de métodos de programação inteira, visto que através da revisão bibliográfica este método apresentou consideráveis resultados.

Assim como os modelos anteriores apresentaram diversas incongruências com a realidade prática da universidade estudada, é esperado que este trabalho acabe por trilhar o mesmo caminho, visto que o problema em questão realmente apresenta grande parte de sua complexidade no entendimento e modelagem de como as diversas partes da instituição interagem entre si, porém, espera-se que este documento possa servir como uma boa base para o entendimento de sua estrutura.

Quanto ao software, mesmo que o prioritário seja a sua funcionalidade, é esperado que o seu design seja o mais intuitivo, fluido e prático quanto for possível. Sendo esta tarefa direcionada mais à experiência do usuário, possivelmente tangenciando o problema central de construção de grades horárias.

Considerando que as duas tentativas anteriores resultaram em métodos que embora atingissem seu objetivo, não foram implementados na prática, tem-se como esperado que o mesmo ocorra com este trabalho. Com isso, espera-se que em trabalhos futuros se estude e analise os motivos de falha do uso prático do atual sistema.
