\chapter{Introdução} \label{chap:introducao}             % ## 1

No ensino superior brasileiro, cada curso de uma instituição de ensino tem em seu projeto pedagógico, ou seja, no documento que rege quais as atribuições e justificativas de existência do curso, uma listagem de disciplinas a serem ministradas em cada semestre ao longo de sua duração esperada. Disciplinas estas que para serem cursadas os discentes precisam cumprir determinados requisitos.

Embora haja o planejamento de duração do curso, diversos fatores podem influenciar a previsão, dentre eles podemos citar eventos como: quebra de pré-requisitos, trancamento de matrícula, transferências, reprovações, indisponibilidade de professores, greves, dentre tantos outros.

Estes eventos tendem a, no geral, aumentar o tempo médio para conclusão do curso. Situação em sua maioria indesejada tanto pelos alunos, que mesmo durante seu estudo já visam o mercado de trabalho, quanto pelos professores e pelaa instituição, visto que a evasão do ensino superior brasileiro é um problema existente e estudado a fim de ser minimizado.

Com isso, é esperado que a instituição busque alternativas para tornar mais dinâmica e atrativa a experiência dos discentes durante sua jornada. Uma dessas formas é tentando minimizar o impacto que os atrasos na grade causam nos semestres consecutivos. Para isso sendo então necessária uma análise das disciplinas que devem ser oferecidas no próximo semestre, sendo então necessário definir \textbf{quais}, \textbf{quando}, \textbf{onde}, \textbf{por quem} e \textbf{para quem} serão ministradas. Esta tarefa, entretanto, não é trivial.

\section{Problemáticas} \label{sec:Problemáticas}        % ### 1.1

Embora seja um problema atualmente, isso não significa que seja recente. Desde 1978 \cite{BARHAM1978} o termo \textit{timetabling} encontra-se no meio acadêmico como o termo referente ao tabelamento de grade horária, sendo assim, é este o termo que será principalmente utilizado neste trabalho. Neste artigo de 1978 já se propunha uma forma para que se obtivesse um tabelamento otimizado, e demonstrava que o método utilizado gerava bons resultados.

Outra característica é informada por \citeonline{thomas_visualization_2009} que fala sobre a multidimensional do problema de \textit{timetabling}. Por causa dessa questão há uma complexidade elevada para conseguir conceber visual e mentalmente de que forma os dados relacionados ao problema se estruturam, assim dificultando a elaboração de sistemas computacionais que auxiliem nessa tarefa.

Também segundo \citeonline{miranda_udpskeduler_2012}, embora o problema de atribuição de salas não seja novo e tenha extensa literatura a seu respeito, são poucos os que de fato implementaram um sistema para suporte de decisões. Isso se dá por diversos fatores, também listado pelo autor fazendo referência a trabalhos anteriores, sendo alguns deles a resistência organizacional a mudanças e adoção de novas tecnologias, nível de dificuldade do problema, dentre outros.

Algumas outras características que se apresentam como problemas são a falta de otimalidade das grades horárias desenvolvidas em boa parte das instituições de ensino superior e a quantidade de tempo necessária para a criação dessas grades não-ótimas.

Considerando que situações como a descrita acima são passíveis de ocorrer, e que a tarefa de criação de grades horárias é recorrente, um sistema de suporte à decisão que supra às necessidades dos seus usuários se faz necessário.

\section{Hipótese} \label{sec:Hipótese}                  % ### 1.2

Dada as características intrínsecas ao problema de agendamento de grade horária, é esperado que os \textit{softwares} atualmente existentes que lidam com este problema não apresentem completas capacidades de se moldar ao caso de uma instituição específica.

E, caso a primeira hipótese se apresente correta, o \textit{software} a ser desenvolvido, assim como seus similares, se apresentará como uma solução plausível para a resolução do problema proposto embora ainda apresente melhorias possíveis a serem implementadas. O \textit{software} se apresentará de tal forma que os \textit{stakeholders} que, esperadamente, decidirem não o utilizar não causarão a impossibilidade do uso do sistema.

\section{Objetivos} \label{sec:Objetivos}                % ### 1.3

Os objetivos desta monografia podem ser divididos entre gerais e específicos, não havendo relação de superioridade de um em relação ao outro, visto que ambos igualmente nortearão o desenvolvimento da pesquisa.

% \subsection{Gerais} \label{subsec:Gerais}                % #### 1.3.1

Como \textbf{objetivos gerais}, espera-se conseguir desenvolver um sistema de suporte à decisão tal que aumente a eficiência, eficácia e efetividade do processo de criação de grades horárias que semestralmente demandam extensa quantidade de tempo dos coordenadores de curso na UENF e não alcançam a otimalidade. Nesse processo, também é esperado que as grades horárias finais tragam benefícios aos alunos como forma de mais disciplinas à sua disposição. Visto que estes muitas vezes lidam com grades horárias que não contemplam suas reais demandas. Dessa forma aumentando a satisfação de todos os participantes do processo, desde os coordenadores de curso até os alunos.

% \subsection{Específicos} \label{subsec:EspeEspecíficos} % #### 1.3.2

Como \textbf{objetivos mais específicos}, podemos listar os seguintes:

\begin{itemize}
  \item Entender de que forma os setores administrativos da UENF atualmente lidam com a questão do \textit{timetabling};
  \item Obter os aprimoramentos desejados pelos responsáveis na criação de grades horárias;
  \item Modelar o sistema de \textit{timetabling} de acordo com os requisitos demandados;
        % \item Encontrar o que é necessário para a adoção da aplicação de tabelamento de horário;
  \item Incentivar o uso de uma ferramenta centralizada para a otimização do \textit{Timetabling Problem}.
\end{itemize}

\section{Justificativas} \label{sec:Justificativas}      % ### 1.4

Levando em conta a problemática evidenciada e os sucessos prévios dos artigos anteriores, vê-se grande potencial de auxílio e aumento na satisfação de todos os que utilizarem os métodos propostos. Não havendo um sistema geral que solucione todos os casos como evidenciado pelos pesquisadores da área, resta aos interessados rumarem em busca de uma solução entalhada nos moldes de sua instituição específica. Considerando que é um problema existente atualmente e que uma solução está disponível, o que se torna necessário é realizar o esforço inicial suficiente para que ocorra a quebra da inércia em que se encontram os processos ineficientes usuais para assim alcançar um melhor. Sendo assim, faz-se válida a pesquisa e desenvolvimento de um \textit{software} que vise tal propósito.

\section{Metodologia} \label{sec:Metodologia}            % ### 1.5

Considerando as dificuldades encontradas em trabalhos anteriores, entende-se que o maior desafio será superar as especificidades que serão encontradas durante a modelagem da universidade em questão. Para isso, será inicialmente necessária uma pesquisa bibliográfica com foco no estudo das abordagens qualitativas realizadas anteriormente que obtiveram sucesso em elicitar os requisitos adequados para as instituições de ensino.

Com este conhecimento, um material inicial para a pesquisa exploratória e qualitativa deve ser desenvolvido levando em conta as questões próprias da universidade em questão, visando também coletar dados relevantes para uma futura pesquisa com maior enfoque em características emergentes que a pesquisa anterior pode levantar, similar a como foi proposto e realizado por \citeonline{Andre2018}. Esta pesquisa exploratória sobre a realidade da instituição se subdivide em três frentes: o estudo sobre a \textbf{documentação teórica}, \textbf{entrevista} com os \textit{stakeholders} e \textbf{formulário} direcionado aos alunos.

% Na \textbf{documentação teórica}, espera-se encontrar informações sobre a estrutura organizacional da UENF, bem como as regras que regem a criação de grades horárias. Quais são os cargos envolvidos e quais são as responsabilidades de cada um deles. Com esta informação, estabelecem-se assim os \textit{stakeholders} iniciais. Na \textbf{entrevista}, algumas informações esperadas giram em torno das percepções dos \textit{stakeholders} do sistema proposto. Estas percepções incluem o entendimento deles quanto ao método atual e às alternativas existentes, nível de insatisfação com o método atual, nível de desejo quanto à um novo método. Além disso, espera-se aproveitar o ensejo para elicitar as características e funcionalidades que gostariam de ter em um sistema de suporte à decisão, solicitando também que deem informações adicionais que gostariam de acrescentar. \textbf{Questionamentos} também serão realizados com alunos, porém em formato de formulário online para atingir mais objetivamente uma quantidade mais vasta de respondentes. Espera-se obter informações sobre a satisfação dos alunos com as grades horárias atuais, para assim poder confirmar se a insatisfação é generalizada e percebida por todos os envolvidos.

Sendo compreendido então o cenário atual da universidade, será então necessário modelar o sistema de suporte à decisão de acordo com as especificidades encontradas. Por fim, será apresentado o processo do desenvolvimento do sistema, quais foram as suas versões, quais funcionalidades foram desenvolvidas e quais foram as tecnologias utilizadas.

\section{Organização} \label{sec:Organização}            % ### 1.6

Este trabalho abordará capítulos que de forma resumida lidam com os seguintes tópicos:

\begin{itemize}
  \item O \autoref{chap:introducao} na \autopageref{chap:introducao} de introdução traça informações gerais sobre o assunto do trabalho, elaborando mais detalhadamente quanto à sua \hyperref[sec:Problemáticas]{problemática}, \hyperref[sec:Hipótese]{hipótese}, \hyperref[sec:Objetivos]{objetivos}, \hyperref[sec:Justificativas]{justificativas}, a \hyperref[sec:Metodologia]{metodologia escolhida} e a \hyperref[sec:Organização]{organização de suas informações};
  \item O \autoref{chap:contexto} na \autopageref{chap:contexto} de revisão literária informa mais detalhadamente sobre os problemas de agendamento, suas categorias, soluções, desafios e definições de termos;
  \item O \autoref{chap:instituicao} na \autopageref{chap:instituicao} de contexto da instituição apresenta a \hyperref[sec:estatuto]{Universidade Estadual do Norte Fluminense Darcy Ribeiro (UENF), suas características, estrutura organizacional}, \hyperref[sec:entrevistas]{entrevistas com os \textit{stakeholders} relacionados à criação de grades horárias} e a \hyperref[sec:sequencia]{sequência de passos para a criação de uma grade horária};
  \item O \autoref{chap:modelagem} na \autopageref{chap:modelagem} de modelagem, apresenta-se a conceitualização macro de como o sistema deve se comportar, quais são as suas funcionalidades e quais são os seus objetivos;
  \item O \autoref{chap:desenvolvimento} na \autopageref{chap:desenvolvimento} de desenvolvimento, pode ser visto o processo de criação do sistema, partindo dos projetos anteriores, passando pela prototipação, para então chegar ao desenvolvimento em três etapas do sistema final;
        % \item O \autoref{chap:experimentos} na \autopageref{chap:experimentos} de experimentos apresenta a validação do sistema desenvolvido através de testes de criação de grade horária seguido de resolução dos conflitos encontrados;
  \item O \autoref{chap:resultados} na \autopageref{chap:resultados} de resultados demonstra um apanhado geral do que foi aprendido e que se obteve como produto final da elaboração deste trabalho;
  \item O \autoref{chap:conclusoes} na \autopageref{chap:conclusoes} da conclusão e trabalhos futuros finaliza o presente trabalho com os pensamentos gerais sobre a pesquisa e desenvolvimento, apresentando as características não abordadas e indicando caminhos a serem seguidos por pesquisadores posteriormente.
  \item O \autoref{chap:apendice} na \autopageref{chap:apendice} de apêndices apresenta informações adicionais que não se encaixam nos capítulos anteriores, mas que são relevantes para o entendimento do trabalho, sendo eles o código-fonte do sistema e da elaboração desse texto, a pesquisa realizada e a análise de seus resultados.
\end{itemize}
