\chapter{Resultados} \label{chap:resultados} % ### 7.

% O QUE DEVERIA TER INTERVALO É ENTRE O FIM DAS INSCRIÇÕES E O INÍCIO DAS AULAS

% adicionar também a possibilidade da mudança informal do horário

\section{Sistema} % ### 7.1.

O sistema desenvolvido atualmente pode ser acessado através \href{https://jvfd3.github.io/timetabling-UENF/}{deste link} \url{https://jvfd3.github.io/timetabling-UENF/}.

\section{Solução Ótima} % ### 7.2.

\section{Alternativas Burocráticas} % ### 7.2.

Além da busca pela solução ótima, o presente trabalho também se propõe a buscar métodos ainda mais alternativos para se amenizar a problemática abordada. Sendo, de forma simples, o uso de meios burocráticos disponíveis na instituição que abre alguns caminhos para uma maior praticidade no processo de resolução do problema. Entretanto, é necessário que se tenha em mente que a burocracia é um processo lento e que pode ser desgastante, sendo até mesmo esperado que não seja desejado por parte dos construtores da grade horária.

Essas alternativas não geram por si só uma solução para o problema, em termos metafóricos, se o sistema é a engrenagem que faz a máquina funcionar, as alternativas burocráticas são o óleo que pode fazer a engrenagem funcionar de forma mais suave, mesmo que não seja estritamente necessário.

\subsection{Tempo de elaboração das grades} % #### 7.1.1.

Durante as entrevistas do \autoref{chap:instituicao} da \autoref{sec:entrevistas}, uma alternativa válida para a amenização da problemática abordada é a alteração do calendário anual da UENF que define férias de duas semanas entre os semestres. Caso seu calendário seja alterado para que as férias sejam de três semanas, o problema de agendamento teria maior tempo para ser resolvido, assim fazendo com que a solução ótima seja provável de ser alcançada.

Mais especificamente: no atual semestre (2024.1), o encerramento do semestre está previsto para o dia 6 de julho. Já o prazo limite para o cadastro de novas

% Abreviados
\begin{table}[H] \centering \caption{Calendário Acadêmico da SECACAD de 2023.1 (simplificado para mostrar apenas os eventos relevantes)} \label{tab:calendario_SECACAD-2023.1}
  \begin{tabular}{| l r |}
    \hline
    \textbf{Atividades}                                              & \textbf{Data} \\
    \hline
    Prazo limite de cad. de nov. discip. a serem ofer. em 2023.1     & até 20/01     \\
    Prazo limite para criação de turmas a serem oferecidas em 2023.1 & 20/01 a 15/02 \\
    Renovação de matrícula de 2023.1                                 & 28/02 a 03/03 \\
    Início do período letivo de 2023.1                               & 06/03         \\
    Inclusão e exclusão de disciplinas                               & 06/03 à 20/03 \\
    Encerramento do período letivo de 2023.1                         & 07/07         \\
    Prazo limite: Entrega dos resultados à SECACAD                   & 14/07         \\
    \hline
  \end{tabular}
\end{table}

% Citar o Estatuto da forma correta.

Segundo o Artigo 28 do Estatuto da UENF, compete à secretaria acadêmica a elaboração da proposta de calendário escolar para que seja aprovado pelo Colegiado Acadêmico. Enquanto que o Artigo 63 da seção 2 do capítulo 1, informa que os calendários do curso de graduação devem ser aprovados pelas correspondentes câmaras, com observância do calendário da universidade.
\begin{comment}
\begin{table}[H] \centering
  \caption{Calendário Acadêmico da SECACAD de 2023.1 (simplificado para mostrar apenas os eventos relevantes)}
  \label{tab:calendario_SECACAD-2023.1}
  \begin{tabular}{| l r |}
    \hline
    \textbf{Atividades}                                                                           & \textbf{Data}           \\
    \hline
    Prazo limite de cadastro de novas disciplinas a serem oferecidas no 1º período letivo de 2023 & até 20/01/2023          \\
    Prazo limite para criação de turmas a serem oferecidas no 1º período letivo de 2023           & 20/01/2023 a 15/02/2023 \\
    Renovação de matrícula do 1º período letivo/2023                                              & 28/02 a 03/03           \\
    Início do 1º período letivo de 2023                                                           & 06/03                   \\
    Inclusão e exclusão de disciplinas                                                            & 06/03 à 20/03           \\
    Encerramento do 1º período letivo de 2023                                                     & 07/07                   \\
    \hline
  \end{tabular}
\end{table}

Logo, quanto à alteração do calendário acadêmico, a alteração mostra-se como possível, sendo necessário apenas que o processo burocrático necessário seja enfrentado.
\begin{table}[H] \centering
  \caption{Calendário Acadêmico da SECACAD de 2023.2 (simplificado para mostrar apenas os eventos relevantes)}
  \label{tab:calendario_SECACAD-2023.2}
  \begin{tabular}{| l r |}
    \hline
    \textbf{Atividades}                                                                           & \textbf{Data} \\
    \hline
    Prazo limite de cadastro de novas disciplinas a serem oferecidas no 2º período letivo de 2023 & até 14/07     \\
    Prazo limite: Entrega dos resultados à SECACAD                                                & 14/07         \\
    Prazo limite para criação de turmas a serem oferecidas no 2º período letivo de 2023           & 17 a 28/07    \\
    Renovação de matrícula do 2º período letivo/2023                                              & 01/08 a 04/08 \\
    Início do 2º período letivo de 2023                                                           & 07/08         \\
    Inclusão e exclusão de disciplinas                                                            & 14 a 21/08    \\
    Encerramento do 2º período letivo de 2023                                                     & 08/12         \\
    Prazo limite: Entrega dos resultados à SECACAD                                                & 15/12         \\
    \hline
  \end{tabular}
\end{table}
\end{comment}

\subsection{Alteração forçada de horários} % #### 7.1.2.
\begin{table}[H] \centering \caption{Calendário Acadêmico da SECACAD de 2023.2 (simplificado para mostrar apenas os eventos relevantes)} \label{tab:calendario_SECACAD-2023.2}
  \begin{tabular}{| l r |}
    \hline
    \textbf{Atividades}                                              & \textbf{Data} \\
    \hline
    Prazo limite de cad. de nov. discip. a serem ofer. em 2023.2     & até 14/07     \\
    Prazo limite para criação de turmas a serem oferecidas em 2023.2 & 17 a 28/07    \\
    Renovação de matrícula de 2023.2                                 & 01/08 a 04/08 \\
    Início do período letivo de 2023.2                               & 07/08         \\
    Inclusão e exclusão de disciplinas                               & 14 a 21/08    \\
    Encerramento do período letivo de 2023.2                         & 08/12         \\
    Prazo limite: Entrega dos resultados à SECACAD                   & 15/12         \\
    \hline
  \end{tabular}
\end{table}

Segundo o parágrafo primeiro do artigo 36 das Normas de Graduação, ``qualquer alteração de horário/turno após o período de matrícula deverá ter a anuência por escrito de todos os discentes matriculados na turma''. Seguindo ao segundo parágrafo do mesmo artigo, temos que ``a alteração de horário das aulas da turma deverá ter a anuência da Coordenação de Curso e a ciência do Chefe do Laboratório responsável pela disciplina''.

Outra alternativa que aproveita de uma brecha nas normas supracitadas é a possibilidade de se criar novas turmas para as disciplinas que possuem horários conflitantes, assim direcionando os alunos para que se desinscrevam da turma anterior.
\begin{table}[H] \centering \caption{Calendário Acadêmico de 2023 (simplificado para mostrar apenas os eventos relevantes)} \label{tab:calendario_2023-Enxuto}
  \begin{tabular}{| c r l c |}
    \hline
    \textbf{Vigência} & \textbf{Fase}         & \textbf{Atividades}                              & \textbf{Data} \\
    \hline
    % 2022.2          & \textbf{Fim}          & período letivo                                   & 14/12/22      \\
    % 2022.2          & \textbf{\textit{Fim}} & entrega dos resultados à SECACAD                 & 21/12/22      \\

    2023.1            & \textbf{\textit{Fim}} & cadastro de novas disciplinas a serem oferecidas & 20/01/23      \\
    2023.1            & \textbf{Início}       & criação de turmas a serem oferecidas             & 20/01/23      \\
    2023.1            & \textbf{Fim}          & criação de turmas a serem oferecidas             & 15/02/23      \\
    2023.1            & \textbf{Início}       & renovação de matrícula                           & 28/02/23      \\
    2023.1            & \textbf{Fim}          & renovação de matrícula                           & 03/03/23      \\
    2023.1            & \textbf{Início}       & período letivo                                   & 06/03/23      \\
    2023.1            & \textbf{Início}       & inclusão e exclusão de disciplinas               & 06/03/23      \\
    2023.1            & \textbf{Fim}          & inclusão e exclusão de disciplinas               & 20/03/23      \\
    2023.1            & \textbf{Fim}          & período letivo                                   & 07/07/23      \\
    2023.1            & \textbf{\textit{Fim}} & entrega dos resultados à SECACAD                 & 14/07/23      \\

    2023.2            & \textbf{\textit{Fim}} & cadastro de novas disciplinas a serem oferecidas & 14/07/23      \\
    2023.2            & \textbf{Início}       & criação de turmas a serem oferecidas             & 17/07/23      \\
    2023.2            & \textbf{Fim}          & criação de turmas a serem oferecidas             & 28/07/23      \\
    2023.2            & \textbf{Início}       & renovação de matrícula                           & 01/08/23      \\
    2023.2            & \textbf{Fim}          & renovação de matrícula                           & 04/08/23      \\
    2023.2            & \textbf{Início}       & período letivo                                   & 07/08/23      \\
    2023.2            & \textbf{Início}       & inclusão e exclusão de disciplinas               & 14/08/23      \\
    2023.2            & \textbf{Fim}          & inclusão e exclusão de disciplinas               & 21/08/23      \\
    2023.2            & \textbf{Fim}          & período letivo                                   & 08/12/23      \\
    2023.2            & \textbf{\textit{Fim}} & entrega dos resultados à SECACAD                 & 15/12/23      \\

    % 2024.1          & \textbf{Início}       & renovação de matrícula                           & 26/02/24      \\
    % 2024.1          & \textbf{Fim}          & renovação de matrícula                           & 01/03/24      \\
    % 2024.1          & \textbf{Início}       & período letivo                                   & 04/03/24      \\
    \hline
  \end{tabular}
\end{table}

Sendo assim possível alterar os horários de aula caso seja necessária para que haja uma melhora geral na distribuição das turmas na grade horária, mais uma vez sendo necessário superar o processo burocrático necessário.

\newcommand{\altered}{\cellcolor[HTML]{79b8ff}} % Define a cor desta célula
\newcommand{\removeLine}{\rowcolor[HTML]{b31d28}} % Define a cor da próxima linha
\newcommand{\addLine}{\rowcolor[HTML]{22863a}} % Define a cor da próxima linha

\begin{table}[H] \centering \caption{Calendário Acadêmico de 2023 - Aprimorado} \label{tab:calendario_2023-Aprimorado}
  \begin{tabular}{| c r l r |}
    \hline
    \textbf{Vigência} & \textbf{Fase}         & \textbf{Atividades}                & \textbf{Data}                  \\
    \hline
    2023.1            & \textbf{\textit{Fim}} & cadastro de novas disciplinas      & \sout{20/01} 13/01/23 \altered \\
    2023.1            & \textbf{Início}       & criação de turmas                  & \sout{20/01} 13/01/23 \altered \\ \removeLine
    \sout{2023.1}     & \sout{\textbf{Fim}}   & \sout{criação de turmas}           & \sout{15/02/23}                \\
    2023.1            & \textbf{Início}       & renovação de matrícula             & \sout{28/02} 21/02/23 \altered \\
    2023.1            & \textbf{Fim}          & renovação de matrícula             & \sout{03/03} 24/02/23 \altered \\
    2023.1            & \textbf{Início}       & período letivo                     & \sout{06/03} 27/02/23 \altered \\
    2023.1            & \textbf{Início}       & inclusão e exclusão de disciplinas & \sout{06/03} 27/02/23 \altered \\
    2023.1            & \textbf{Fim}          & inclusão e exclusão de disciplinas & \sout{20/03} 13/03/23 \altered \\ \addLine
    2023.1            & \textbf{Fim}          & criação de turmas                  & 20/03/23                       \\
    2023.1            & \textbf{Fim}          & período letivo                     & \sout{07/07} 30/06/23 \altered \\
    2023.1            & \textbf{\textit{Fim}} & entrega dos resultados à SECACAD   & \sout{14/07} 07/07/23 \altered \\

    2023.2            & \textbf{\textit{Fim}} & cadastro de novas disciplinas      & 14/07/23                       \\
    2023.2            & \textbf{Início}       & criação de turmas                  & 17/07/23                       \\ \removeLine
    \sout{2023.2}     & \sout{\textbf{Fim}}   & \sout{criação de turmas}           & \sout{28/07/23}                \\
    2023.2            & \textbf{Início}       & renovação de matrícula             & 01/08/23                       \\
    2023.2            & \textbf{Fim}          & renovação de matrícula             & 04/08/23                       \\
    2023.2            & \textbf{Início}       & período letivo                     & 07/08/23                       \\
    2023.2            & \textbf{Início}       & inclusão e exclusão de disciplinas & 14/08/23                       \\
    2023.2            & \textbf{Fim}          & inclusão e exclusão de disciplinas & 21/08/23                       \\ \addLine
    2023.2            & \textbf{Fim}          & criação de turmas                  & 21/08/23                       \\
    2023.2            & \textbf{Fim}          & período letivo                     & 08/12/23                       \\
    2023.2            & \textbf{\textit{Fim}} & entrega dos resultados à SECACAD   & 15/12/23                       \\
    \hline
  \end{tabular}
\end{table}
