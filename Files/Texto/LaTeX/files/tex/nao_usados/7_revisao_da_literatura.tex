\chapter{Revisão da literatura}

Alguns pontos relevantes a se aprofundar com a revisão literária são:

\section{Qual é o problema?}

% [Usar a fonte para explicar o problema de timetabling: https://sci-hub.se/https://doi.org/10.1007/978-3-030-33720-9_21]
O ``problema de organização de grade horária'' (\textit{timetabling problem}) é uma subcategoria do problema de \textbf{agendamento} (\textit{scheduling}) que por sua vez é definido por \cite{WREN1996} como sendo:

\begin{quote}\footnotesize
  Resolver problemas práticos relacionados à alocação, sujeito a restrições, de recursos a objetos sendo colocados no espaço-tempo, usando ou desenvolvendo quaisquer ferramentas que possam ser apropriadas. Os problemas irão frequentemente se relacionar à satisfação de certos objetivos.
\end{quote}

Wren também informa que uma determinada \textit{timetable} apenas informa quando eventos serão realizados, e esse conceito se encaixa no que proponho neste documento. Entretanto, ele também diz que ``a grade horária de classes de uma universidade deve levar em conta a disponibilidade de professores individualmente.'', bem como a atribuição de salas, sendo ambos limitados por restrições duras e flexíveis (\textit{hard and soft constraints}).

\section{Há solução?}

Existem diversas implementações já realizadas, utilizando de uma variedade de métodos. Em seu trabalho, J. Miranda \cite{MIRANDA2012505} informa sobre diversos sistemas baseados em computador para auxiliar na tarefa de agendamento. J. Miranda também cita um dos métodos de resolução como sendo o \textbf{modelo de programação inteira} e \textbf{heurísticas}.

% Já [Vinod](\ref{https://www.researchgate.net/publication/301564510_ACADEMIC_TIMETABLE_SCHEDULING_REVISITED}) cita diversos outros autores como: .[34](a) [64](a)D.Abramson(1991), D.Abramson et al (1999) proposed simulated annealing [66](a) [20](a) M. Wright(1996) used heuristic search and A. Schaerf (1999) used Local search techniques for School.

% Preciso formatar muito direitinho isso daqui. Tem qualidade de referência, mas tá porco o uso

\section{Quais são os desafios da solução do problema?}

Também segundo J. Miranda, embora o problema de atribuição de salas não seja novo e tenha extensa literatura a seu respeito, são poucos os que de fato implementaram um sistema para suporte de decisões. Isso se dá por diversos fatores, também listado pelo autor fazendo referência a trabalhos anteriores, sendo alguns deles a resistência organizacional à mudanças e adoção de novas tecnologias, nível de dificuldade do problema, dentre outros.

% Pegar a referência original?

Outra característica é informada por Joshua \cite{THOMAS2009} que fala sobre a multidimensional do problema de \textit{timetabling}. Por causa dessa questão há uma complexidade elevada para conseguir conceber visual e mentalmente de que forma os dados relacionados ao problema se estruturam, assim dificultando a elaboração de sistemas computacionais que auxiliem nessa tarefa.

Tomáš Müller \cite{MURRAY2007} traz a questão da modelagem como um dos maiores obstáculos. À medida em que a complexidade aumenta, se torna cada vez mais difícil desenvolver uma solução efetiva. Assim fazendo com que a solução para uma universidade possa não ter utilidade para outras, ou até mesmo não seja capaz de lidar com todos os problemas de uma mesma universidade.

\section{Como resolver os desafios?}

Apesar do contra-fluxo encontrado na resolução dessa problema, Tomáš cita que, apesar da complexidade, é sim possível desenvolver soluções que tenham uso prático, mesmo que não seja um processo fácil. As ferramentas existem e estão disponíveis. Restando então considerar e resolver as preocupações dos usuários às questões além dos métodos de resolução.

Com isso, entramos também no ramo da Interação Homem-Máquina, ramo abordado por Dinata \cite{ANDRE2018} que visou em seu desenvolvimento a criação de uma interface focada no usuário. Assim minimizando o atrito na abordagem desse problema complexo.
